% included by stdcommon.tex

\newcommand{\latexhtmlLARGE}{\normalsize} % e.g. /k/l/sasp/summary.tex

%\newcommand{\twidth}{{ 4.3in}} % USE INSTEAD OF TEXTWIDTH FOR HTML WIDTH
%\newlength{\twidth}\setlength{\twidth}{4.1in} % was 4.3 until 5/14/2008
\newlength{\twidth}\setlength{\twidth}{6in} % was 4.1 until 11/29/2010
%  8in should nearly fill default HTML pageview margin, but then PASP gets
%  overfull vboxes in images.tex (265.5 pts over was the worst ~265.5/72~3.7in)
% Based on iPad, iPhone, 
\newlength{\theight}\setlength{\theight}{0.75\twidth} % A 16"W x 12"H monitor = 4/3 width/height ratio
\newcommand{\captionwidth}{\twidth} % no \textwidth in HTML environment
\newcommand{\mlfigwidth}{\twidth} % matlab figures too large initially

% See stdfigs-latex.tex for explanation:
\input{jos-latex/styles/mycaptionparlabel.tex}
% Should pull in jos-latex/styles/mycaptionparlabel-html.tex

\newcommand{\stdFigScale}{0.8} % Good full-page scaling for Matlab plots
\newcommand{\stdSmallFigScale}{0.5} % Good for little by-the-way figures
% Used in bbt.tex, thms.tex

% =================== POSTSCRIPT FIGURE SUPPORT ==================

% Usage: \myFigure{num}{caption}
\newcommand{\myFigure}[2]{
	\begin{figure}[htbp]
	\includegraphics{eps/#1} % the figure
	%\caption{#2}\label{fig:#1}
        \centering
        \mycaptionparlabel{#2}{fig:#1}
	\end{figure}
}

% Usage: \myFigureNoCap{num}
\newcommand{\myFigureNoCap}[1]{
	\begin{figure}[htbp]
	\includegraphics{eps/#1} % the figure
	\end{figure}
}

% Usage: \myFigureToWidth{name}{width}{caption}
\newcommand{\myFigureToWidth}[3]{
	\begin{figure}[htbp]
% Cannot use \resizebox to make figures larger than natural size.
% latex2html / graphicx.perl will cut them off on the right and top.
%	\resizebox{#2}{!}{\includegraphics{eps/#1}}
	\includegraphics[width=#2]{eps/#1}
	%\caption{#3}\label{fig:#1}
        \mycaptionparlabel{#3}{fig:#1}
	\end{figure}
}

% Usage: \myFigureToWidthNoCap{name}{width}
\newcommand{\myFigureToWidthNoCap}[2]{
	\begin{figure}[htbp]
% Cannot use \resizebox to make figures larger than natural size.
% latex2html / graphicx.perl will cut them off on the right and top.
%	\resizebox{#2}{!}{\includegraphics{eps/#1}}
%????	Gets a divide-by-zero crash for /w/aspf/
%       \includegraphics[width=#2]{eps/#1}
	\includegraphics{eps/#1}
	\end{figure}
}

% Usage: \myFigureRotateToWidth{name}{angle}{width}{caption}
\newcommand{\myFigureRotateToWidth}[4]{
	\begin{figure}[htbp]
% FIXME: rotation AND scaling fails in /l/l2h/styles/graphics-support.perl
% Therefore, we continue to use resizebox which cannot work
%orig:	\rotatebox{#2}{\resizebox{#3}{!}{\includegraphics{eps/#1}}}
%new:	\includegraphics[angle=#2,width=#3]{eps/#1}
%kludge:
	\includegraphics[width=#3]{eps/#1}
	\caption{#4 --- [FIXME: Cannot do both rotation and scaling in latex2html graphicx support]}\label{fig:#1}
	\end{figure}
}

% Usage: \myFigureRotateToBox{name}{angle}{width}{height}{caption}
\newcommand{\myFigureRotateToBox}[5]{
	\begin{figure}[htbp]
% FIXME: rotation AND scaling fails in /l/l2h/styles/graphics-support.perl
% Therefore, we continue to use resizebox which cannot work
%orig:	\rotatebox{#2}{\resizebox{#3}{!}{\includegraphics{eps/#1}}}
%new:	\includegraphics[angle=#2,width=#3]{eps/#1}
%kludge:
	\includegraphics[width=#3,height=#4]{eps/#1}
	\caption{#5 --- [FIXME: Cannot do both rotation and scaling in latex2html graphicx support]}\label{fig:#1}
	\end{figure}
}

% Usage: \myFigureToBox{name}{width}{height}{caption}
\newcommand{\myFigureToBox}[4]{ 
	\begin{figure}[htbp]
%prv:	\resizebox{#2}{#3}{\includegraphics{eps/#1}}
	\includegraphics[width=#2,height=#3]{eps/#1}
	%\caption{#4}\label{fig:#1}
        \mycaptionparlabel{#4}{fig:#1}
	\end{figure}
}

% Usage: \myFigureScale{name}{scaleFactor}{caption}
\newcommand{\myFigureScale}[3]{ 
	\begin{figure}[htbp]
%prv:	\scalebox{#2}{\includegraphics{eps/#1}}
	\includegraphics[scale=#2]{eps/#1}
	%\caption{#3}\label{fig:#1}
        \mycaptionparlabel{#3}{fig:#1}
	\end{figure}
}

\newcommand{\myFigureHere}[2]{\myFigure{#1}{#2}}

% Usage: \myTwoFiguresToWidth{name1}{name2}{width}{caption1}{caption2}{caption}
\newcommand{\myTwoFiguresToWidth}[6]{
	\begin{figure}[htbp]
	\centering
% Cannot use \resizebox to make figures larger than natural size.
% latex2html / graphicx.perl will cut them off on the right and top.
%  	   \resizebox{#3}{!}{\includegraphics{eps/#1}}
\begin{rawhtml}<HR>\end{rawhtml}
  	   \includegraphics{eps/#1}
	   \centerline{#4} % subcaption
  	   \label{fig:#1}
\begin{rawhtml}<HR 50\%>\end{rawhtml}
	\\
%  	   \resizebox{#3}{!}{\includegraphics{eps/#2}}
 	   \includegraphics{eps/#2}
	   \centerline{#5} % subcaption
  	   \label{fig:#2}
\begin{rawhtml}<HR 50\%>\end{rawhtml}
	\caption{#6}
	\end{figure}
}

% Usage: \myTwoFiguresToBoxes{name1}{name2}{width}{sliceheight}{caption1}{caption2}{caption}
\newcommand{\myTwoFiguresToBoxes}[7]{
	\begin{figure}[htbp]
	\centering
% Cannot use \resizebox to make figures larger than natural size.
% latex2html / graphicx.perl will cut them off on the right and top.
%  	   \resizebox{#3}{!}{\includegraphics{eps/#1}}
        \includegraphics{eps/#1}\\
	\centerline{#5} % subcaption
        \label{fig:#1}
	\\
% 	   \resizebox{#3}{#4}{\includegraphics{eps/#2}}
        \includegraphics{eps/#2}\\
	\centerline{#6} % subcaption
        \label{fig:#2}
	\caption{#7}
	\end{figure}
}

% Usage: \myThreeFiguresToWidth{name1}{name2}{name3}{width}{caption1}{caption2}{caption3}{caption}
\newcommand{\myThreeFiguresToWidth}[8]{
	\begin{figure}[htbp]
	\centering
% Cannot use \resizebox to make figures larger than natural size.
% latex2html / graphicx.perl will cut them off on the right and top.
%  	   \resizebox{#4}{!}{\includegraphics{eps/#1}}
\begin{rawhtml}<HR>\end{rawhtml}
  	   \includegraphics{eps/#1}
	   \centerline{#5} % subcaption
  	   \label{fig:#1}
\begin{rawhtml}<HR 50\%>\end{rawhtml}
	\\
%  	   \resizebox{#4}{!}{\includegraphics{eps/#2}}
 	   \includegraphics{eps/#2}
	   \centerline{#6} % subcaption
  	   \label{fig:#2}
\begin{rawhtml}<HR 50\%>\end{rawhtml}
	\\
%  	   \resizebox{#4}{!}{\includegraphics{eps/#3}}
 	   \includegraphics{eps/#3}
	   \centerline{#7} % subcaption
  	   \label{fig:#2}
\begin{rawhtml}<HR 50\%>\end{rawhtml}
	\caption{#8}
	\end{figure}
\begin{rawhtml}<HR>\end{rawhtml}
}

% Usage: \myThreeFiguresToBoxes{name1}{name2}{name3}{width}{sliceheight}{caption1}{caption2}{caption3}{caption}
\newcommand{\myThreeFiguresToBoxes}[9]{
  \myThreeFiguresToWidth{#1}{#2}{#3}{#4}{#6}{#7}{#8}{#9}
}

% Usage: \myFourFiguresToWidth{name1}{name2}{name3}{name4}{width}{caption}
\newcommand{\myFourFiguresToWidth}[6]{
	\begin{figure}[htbp]
	\centering
\begin{rawhtml}<HR>\end{rawhtml}
% Cannot use \resizebox to make figures larger than natural size.
% latex2html / graphicx.perl will cut them off on the right and top.
%  	   \resizebox{#3}{!}{\includegraphics{eps/#1}}
  	   \includegraphics{eps/#1}
  	   \includegraphics{eps/#2}
\begin{rawhtml}<HR 50\%>\end{rawhtml}
	\\
 	   \includegraphics{eps/#3}
 	   \includegraphics{eps/#4}
\begin{rawhtml}<HR 50\%>\end{rawhtml}
	\caption{#6}
        \label{figboth:#1}
	\end{figure}
}

\newcommand{\myFourFiguresToWidths}[9]{
% Individual widths ignored in HTML version (\resizebox gets cut off):
myFourFiguresToWidth{#1}{#2}{#3}{#4}{#5}{#9}
}

%======================== TABLE SUPPORT ========================

% Usage: \myTable{Num[table prefix added]}{MultiLineCaption.}{table}
\newcommand{\myTable}[3]{
	\begin{table}[htbp]
	\begin{center}
	#3
	\end{center}
	\caption{#2}
	\label{table:#1} % MUST GO AFTER CAPTION OR INSIDE ITS ARGUMENT
	\end{table}
}

% Usage: \myTableHere{MultiLineCaption.}{table}
\newcommand{\myTableHere}[3]{
	\begin{table}[h]
	\begin{center}
	#2
	\end{center}
	\caption{#1}
	\end{table}
}

% =================== CODE LISTING SUPPORT ===================

% The code env creates a code-listing box.
%USAGE: 
%  \begin{code}{name}{caption} 
%  \begin{verbatim}
%  ... 
%  \end{verbatim}
%  \end{code}

\newenvironment{code}[2]{% HTML VERSION
    \begin{figure}
	\centerline{\hrule\captionwidth}
	\caption{#2}\label{code:#1}
	% Inhibits fusing of figure into image:
	% DO NOT PLACE A COMMENT WITHIN THE MAKEIMAGE ENVIRONMENT (l2h bug)
	\begin{makeimage}
	\end{makeimage}
	}{%
	\centerline{\hrule\captionwidth}
    \end{figure}
} % end code env def

% =================== XFIG FIGURE SUPPORT ===================
%
% Usage: 
%           \myTexFigure{name}{caption}
%
% Figures generated by xfig are assumed to have been exported
% in .latex mode ('both parts').  This means both name.pstex_t 
% (containing a latex picture environment) and name.pstex 
% (containing raw postscript) are written by xfig.  
% The .pstex_t and .pstex files must be in the subdirectory 'fig'
% of the LaTeX document directory.
%
% When exporting figures from xfig, make sure
% to reduce them to the desired size.  There are no resizing
% facilities here, and there can't be because, while postscript
% can be arbitrarily resized, the 'manual' annotations in the
% .pstex_t file cannot.

%Until 2020-08-02:
\usepackage{color} % xfig figures exporting pstex_t files need this
%\usepackage{xcolor} % /w/spectilt/

\newcommand{\myTexFigure}[2]{
  \myFigure{#1}{#2} % We now convert .fig to .eps in Makefile using figtex2eps
}

% Obsolete:
\newcommand{\myTexFigurePRV}[2]{
    \begin{figure}[tb]
    \input{fig/#1.pstex_t}
    %\caption{\protect #2}\label{fig:#1}
    \mycaptionparlabel{#2}{fig:#1}
    \end{figure}
}
