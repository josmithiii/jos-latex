% stdfigs-latex.tex
% included by stdcommon.tex

\newcommand{\latexhtmlLARGE}{\LARGE} % e.g. /k/l/sasp/summary.tex

% USE THIS INSTEAD OF TEXTWIDTH TO AVOID FIGURE CUT-OFF IN L2H
\providecommand{\twidth}{\textwidth} % Alternatively defined in stdfigs-html.tex
\providecommand{\theight}{\textheight} % Alternatively defined in stdfigs-html.tex

%\newcommand{\captionwidth}{0.8\textwidth} % code cut off at book dimensions
\providecommand{\captionwidth}{0.9\textwidth}
%\newcommand{\captionwidth}{\textwidth} % better for final book format?

%\newcommand{\mlfigwidth}{\textwidth} % matlab fig width (too large initially) 
\providecommand{\mlfigwidth}{0.9\textwidth} % matlab fig width (too large initially) 

% *** WARNING *** 
% The argument to caption here and in stdfigs-html.tex must be IDENTICAL
% in order for latex2html to be able to find it and assigned the correct
% figure number.  Reason: the .aux file generated by the -latex case is
% processed by the -html case.  Caption matching is literal, and must match
% exactly.  (No extra '\small' or extra expansion or anything like that.)
% We use liberal springling of '\protect' to inhibit macro expansions as
% much as possible.

%9/10/03/jos: \newcommand{\mycaptionparlabel}[2]{\parbox{\captionwidth}{\caption{\protect #1\vspace{10pt}\label{\protect #2}}}}
%9/11/03/jos: \newcommand{\mycaptionparlabel}[2]{\parbox{\captionwidth}{\caption{\protect #1\label{\protect #2}}}}
\newcommand{\mycaption}[1]{\protect\parbox{\captionwidth}{\protect\small #1}}
\newcommand{\mycaptionpar}[1]{\protect\parbox{\captionwidth}{\caption{\protect\small #1}}}
%6/30/05/jos: Use of the \parbox causes all sorts of crud to be emitted
% with the figure caption, specifically
% ``\relax \fontsize  {9}{11}\selectfont  \abovedisplayskip 8.5\p@ plus3\p@ minus4\p@ \abovedisplayshortskip \z@ plus2\p@ \belowdisplayshortskip 4\p@ plus2\p@ minus2\p@ \def \leftmargin \leftmargini \parsep 4\p@ plus2\p@ minus\p@ \topsep 8\p@ plus2\p@ minus4\p@ \itemsep 4\p@ plus2\p@ minus\p@ {\leftmargin \leftmargini \topsep 4\p@ plus2\p@ minus2\p@ \parsep 2\p@ plus\p@ minus\p@ \itemsep \parsep }\belowdisplayskip \abovedisplayskip \protect''
%
% FIXME: To enable use of \parbox in caption, we need a new environment
%        in html.sty ``latexforhtml'' which somehow creates the .aux file
%        used by l2h.  Right now, there is only one latex case.
%
% SOLUTION: This file gets whacked by make to be what it needs to be:
\input{./mycaptionparlabel.tex}
%Fails (``No number for <caption>'' in l2h):
%\newcommand{\mycaptionparlabel}[2]{\protect\parbox{\captionwidth}{\caption{\protect\small #1\label{#2}}}}

\newcommand{\stdFigScale}{0.8} % Good full-page scaling for Matlab plots
\newcommand{\stdSmallFigScale}{0.5} % Good for little by-the-way figures
% Used in bbt.tex, thms.tex

% Strut used to establish a min caption width and set off caption from text:
\newcommand{\mystrut}{\rule{\captionwidth}{0.1pt}} 
% or \includegraphics{eps/strut.eps} for greater uniformity with HTML case
% ot \centerline{\rule{\captionwidth}{0.1pt}}
	
% =================== POSTSCRIPT FIGURE SUPPORT ==================

% latex: 
\providecommand{\figdir}{eps}
%\providecommand{\figdotext}{.eps} % Fails! Extension must be empty!:
\providecommand{\figdotext}{}
% realsimple:\providecommand{\figdir}{figures}\providecommand{\figdotext}{.eps}
% pdflatex:\providecommand{\figdir}{pdf} \providecommand{\figdotext}{.pdf}

% Usage: \myFigure{name}{caption}
\providecommand{\myFigure}[2]{
	\begin{figure}[htbp]
	\centering
	\includegraphics{\figdir/#1\figdotext}
	\mycaptionparlabel{#2}{fig:#1}
	\mystrut
	\end{figure}
}

% Usage: \myFigureNoCap{name}
\newcommand{\myFigureNoCap}[1]{
	\begin{figure}[htbp]
	\centering
	\includegraphics{\figdir/#1\figdotext}
	\end{figure}
}

% Usage: \myFigureToWidth{name}{width}{caption}
\newcommand{\myFigureToWidth}[3]{
	\begin{figure}[htbp]
	\centering
	\resizebox{#2}{!}{\includegraphics{\figdir/#1\figdotext}}
	\mycaptionparlabel{#3}{fig:#1}
%works (l2h fignums exist): \caption{#3}\label{fig:#1}
	\mystrut
	\end{figure}
}

% Usage: \myFigureToWidthNoCap{name}{width}
\newcommand{\myFigureToWidthNoCap}[2]{
	\begin{figure}[htbp]
	\centering
	\resizebox{#2}{!}{\includegraphics{\figdir/#1\figdotext}}
	\end{figure}
}

% Usage: \myFigureRotateToWidth{name}{angle}{width}{caption}
\newcommand{\myFigureRotateToWidth}[4]{
	\begin{figure}[htbp]
	\centering
%prv:	\rotatebox{#2}{\resizebox{#3}{!}{\includegraphics{\figdir/#1\figdotext}}}
	\resizebox{#3}{!}{\rotatebox{#2}{\includegraphics{\figdir/#1\figdotext}}}
	\mycaptionparlabel{#4}{fig:#1}
	\normalsize
	\end{figure}
}

% Usage: \myFigureRotateTwoToWidth{name1}{name2}{angle}{width}{caption}
\newcommand{\myFigureRotateTwoToWidth}[5]{
	\begin{figure}[htbp]
	\centering
        \minipage{#4}
	\resizebox{0.5 #4}{!}{\rotatebox{#3}{\includegraphics{\figdir/#1\figdotext}}}
	\resizebox{0.5 #4}{!}{\rotatebox{#3}{\includegraphics{\figdir/#2\figdotext}}}
	\mycaptionparlabel{#5}{fig:#1}
	\normalsize
        \endminipage
	\end{figure}
}

% Usage: \myFigureRotateToBox{name}{angle}{width}{height}{caption}
\newcommand{\myFigureRotateToBox}[5]{
	\begin{figure}[htbp]
	\centering
	\rotatebox{#2}{\resizebox{#3}{#4}{\includegraphics{\figdir/#1\figdotext}}}
	\mycaptionparlabel{#5}{fig:#1}
	\normalsize
	\end{figure}
}

% Usage: \myFigureToBox{name}{width}{height}{caption}
\newcommand{\myFigureToBox}[4]{ 
	\begin{figure}[htbp]
	\centering
	\resizebox{#2}{#3}{\includegraphics{\figdir/#1\figdotext}}
	\mycaptionparlabel{#4}{fig:#1}
	\mystrut
	\end{figure}
}

% Usage: \myFigureScale{name}{scaleFactor}{caption}
\newcommand{\myFigureScale}[3]{ 
	\begin{figure}[htbp]
	\centering
	\scalebox{#2}{\includegraphics{\figdir/#1\figdotext}}
	\mycaptionparlabel{#3}{fig:#1}
	\mystrut
	\end{figure}
}

\newcommand{\myFigureHere}[2]{\myFigure{#1}{#2}}

% FIXME: cptex cannot handle more than one figure per \my*Figure*{}

% Usage: \myTwoFiguresToWidth{name1}{name2}{width}{caption1}{caption2}{caption}
\newcommand{\myTwoFiguresToWidth}[6]{ % was ``suddenly broken'' at some point, but seems ok now (latex case at least)
	\begin{figure}[htbp]
	\centering
	\subfigure[#4]{ % Top figure = first figure
  	   \resizebox{#3}{!}{\includegraphics{\figdir/#1\figdotext}}
  	   \label{fig:#1}
	}
	\subfigure[#5]{
  	   \resizebox{#3}{!}{\includegraphics{\figdir/#2\figdotext}}
  	   \label{fig:#2}
	}
	\mycaptionparlabel{\protect #6}{figboth:#1}
	\mystrut
	\end{figure}
}

% Usage: \myTwoFiguresToBoxes{name1}{name2}{width}{sliceheight}{caption1}{caption2}{caption}
\newcommand{\myTwoFiguresToBoxes}[7]{
	\begin{figure}[htbp]
	\centering
	\subfigure[#5]{ % Top figure = first figure
  	   \resizebox{#3}{#4}{\includegraphics{\figdir/#1\figdotext}}
  	   \label{fig:#1}
	}
	\subfigure[#6]{ % Bot figure = 2nd figure
  	   \resizebox{#3}{#4}{\includegraphics{\figdir/#2\figdotext}}
  	   \label{fig:#2}
	}
	\mycaptionparlabel{\protect #7}{figboth:#1}
	\mystrut
	\end{figure}
}

\newcommand{\myTwoFiguresToBoxesFigureLabelsBecomeSectionNumber}[7]{
	\begin{figure}[htbp]
	\centering
%	\subfigure[#5]{ % Top figure = first figure
  	   \resizebox{#3}{#4}{\includegraphics{\figdir/#1\figdotext}}
  	   \label{fig:#1}
%	}
%	\subfigure[#6]{ % Bot figure = 2nd figure
  	   \resizebox{#3}{#4}{\includegraphics{\figdir/#2\figdotext}}
  	   \label{fig:#2}
%	}
	\mycaptionparlabel{\protect #7}{figboth:#1}
	\mystrut
	\end{figure}
}

% Usage: \myThreeFiguresToWidth{name1}{name2}{name3}{width}{caption1}{caption2}{caption3}{overall-caption}
\newcommand{\myThreeFiguresToWidth}[8]{
	\begin{figure}[htbp]
	\centering
	\subfigure[#5]{ % Top figure = first figure
  	   \resizebox{#4}{!}{\includegraphics{\figdir/#1\figdotext}}
  	   \label{fig:#1}
	}
	\subfigure[#6]{
  	   \resizebox{#4}{!}{\includegraphics{\figdir/#2\figdotext}}
  	   \label{fig:#2}
	}
	\subfigure[#7]{
  	   \resizebox{#4}{!}{\includegraphics{\figdir/#3\figdotext}}
  	   \label{fig:#3}
	}
	\mycaptionparlabel{\protect #8}{figboth:#1}
	\mystrut
	\end{figure}
}

% Usage: \myThreeFiguresToBoxes{name1}{name2}{name3}{width}{sliceheight}{caption1}{caption2}{caption3}{overall-caption}
\newcommand{\myThreeFiguresToBoxes}[9]{
	\begin{figure}[htbp]
	\centering
	\subfigure[#6]{ % Top figure = first figure
  	   \resizebox{#4}{#5}{\includegraphics{\figdir/#1\figdotext}}
  	   \label{fig:#1}
	}
	\subfigure[#7]{
  	   \resizebox{#4}{#5}{\includegraphics{\figdir/#2\figdotext}}
  	   \label{fig:#2}
	}
	\subfigure[#8]{
  	   \resizebox{#4}{#5}{\includegraphics{\figdir/#3\figdotext}}
  	   \label{fig:#3}
	}
	\mycaptionparlabel{\protect #9}{figboth:#1}
	\mystrut
	\end{figure}
}

% 9 parameters max => cannot have individual captions without
% some kind of hackery! [can an arg be a parseable macro with args?]
% Usage: \myFourFiguresToWidth{name1}{name2}{name3}{name4}{width}{caption}
% NOTE: width is total width, so it is divided by 2 for each plot
%       (actually some room is left for middle separation)
\newcommand{\myFourFiguresToWidth}[6]{
	\begin{figure}[htbp]
	\centering
	\hbox{
	\subfigure{ % top-left
  	   \resizebox{0.48 #5}{!}{\includegraphics{\figdir/#1\figdotext}}
  	   \label{fig:#1}
	}\hspace{0.3in}
	\subfigure{ % top-right
  	   \resizebox{0.48 #5}{!}{\includegraphics{\figdir/#2\figdotext}}
  	   \label{fig:#2}
	}}\vspace{0.1in}
	\hbox{
	\subfigure{ % bot-left
  	   \resizebox{0.48 #5}{!}{\includegraphics{\figdir/#3\figdotext}}
  	   \label{fig:#3}
	}\hspace{0.3in}
	\subfigure{ % bot-right
  	   \resizebox{0.48 #5}{!}{\includegraphics{\figdir/#4}}
  	   \label{fig:#4}
	}}
	\mycaptionparlabel{\protect #6}{figboth:#1}
	\mystrut
	\end{figure}
}

% Same as above, except widths are individually passed:
% Usage: \myFourFiguresToWidths{name1}{name2}{name3}{name4}{width1}{width2}{width3}{width4}{caption}
\newcommand{\myFourFiguresToWidths}[9]{
	\begin{figure}[htbp]
	\centering
	\hbox{
	\subfigure{ % top-left
  	   \resizebox{#5}{!}{\includegraphics{\figdir/#1\figdotext}}
  	   \label{fig:#1}
	}\hspace{0.3in}
	\subfigure{ % top-right
  	   \resizebox{#6}{!}{\includegraphics{\figdir/#2\figdotext}}
  	   \label{fig:#2}
	}}\vspace{0.1in}
	\hbox{
	\subfigure{ % bot-left
  	   \resizebox{#7}{!}{\includegraphics{\figdir/#3\figdotext}}
  	   \label{fig:#3}
	}\hspace{0.3in}
	\subfigure{ % bot-right
  	   \resizebox{#8}{!}{\includegraphics{\figdir/#4}}
  	   \label{fig:#4}
	}}
	\mycaptionparlabel{\protect #9}{figboth:#1}
	\mystrut
	\end{figure}
}

% =================== JPEG (.jpg) FIGURE SUPPORT - PDFLATEX ONLY ==================

% *** REQUIRES PDFLATEX *** (latex requires bounding box in all figures, which jpg does not provide)

% https://www.simplilearn.com/tutorials/deep-learning-tutorial/perceptron
\providecommand{\myJpgFigure}[2]{
  \begin{figure}[htbp]
  \centering
  \includegraphics{jpg/#1.jpg}
  \mycaptionparlabel{#2}{fig:#1}
  \mystrut
  \end{figure}
}

\providecommand{\myJpgFigureToWidth}[3]{
  \begin{figure}[htbp]
  \centering
  \resizebox{#2}{!}{\includegraphics{jpg/#1.jpg}}
  \mycaptionparlabel{#3}{fig:#1}
%works (l2h fignums exist): \caption{#3}\label{fig:#1}
  \mystrut
  \end{figure}
}

% Usage: \myFigureToWidthNoCap{name}{width}
\newcommand{\myJpgFigureToWidthNoCap}[2]{
	\begin{figure}[htbp]
	\centering
	\resizebox{#2}{!}{\includegraphics{jpg/#1.jpg}}
	\end{figure}
}

%======================== TABLE SUPPORT ========================

% Usage: \myTable{Num[table prefix added]}{MultiLineCaption.}{table}
\newcommand{\myTable}[3]{
	\begin{table}[htbp]
	\centering
%	\begin{center}
	#3
	\mycaptionparlabel{#2}{table:#1}
	\mystrut
%	\end{center}
	\end{table}
}

% Usage: \myTableHere{MultiLineCaption.}{table}
\newcommand{\myTableHere}[2]{
	\begin{table}[h]
	\centering
%	\begin{center}
	#2
	{\small \caption{#1}} % Do NOT place \small inside caption arg
	\mystrut
%	\end{center}
	\end{table}
}

% =================== CODE LISTING SUPPORT ===================

% The code env creates a code-listing box.
%USAGE: 
%  \begin{code}{name}{caption} 
%  \begin{verbatim}
%  ... 
%  \end{verbatim}
%  \end{code}
%
% REQUIRES:
% \usepackage{boxedminipage} % See p. 277 of LaTeX Companion

\newenvironment{code}[2]{%LATEX VERSION
\begin{figure}[htbp]
\centering
\gdef\currLabel{#1}\gdef\currCap{#2}
\setlength{\fboxsep}{3mm}
\begin{boxedminipage}{\captionwidth}
\small  % 9/23/03
}%
{
\end{boxedminipage}
\parbox{\captionwidth}{{\small\caption{\currCap}\label{code:\currLabel}}}
\end{figure}
}

\newenvironment{oldcode}[2]{%LATEX VERSION
\begin{figure}[htbp]
\centering
\gdef\currLabel{#1}\gdef\currCap{#2}
\setlength{\fboxsep}{3mm}
\begin{boxedminipage}{\captionwidth}
}%
{
\hrule
\begin{minipage}{\captionwidth} % same effect as \parbox{\captionwidth}
{\small\caption{\currCap}\label{code:\currLabel}}
\end{minipage}
\end{boxedminipage}
\end{figure}
}

% =================== XFIG FIGURE SUPPORT ===================
%
% Usage: 
%           \myTexFigure{name}{caption}
%
% Figures generated by xfig are assumed to have been exported
% in .latex mode ('both parts').  This means both name.pstex_t 
% (containing a latex picture environment) and name.pstex 
% (containing raw postscript) are written by xfig.  
%
% When exporting figures from xfig, make sure to enlarge or
% reduce them to the desired size.  There are no resizing
% facilities here, and there can't be because, while postscript
% can be arbitrarily resized, the 'manual' annotations in the
% .pstex_t file can only be resized by editing the \unitlength setting
% (within in the picture environment).  I see no way to override
% this setting ``externally''.  Furthermore, if magnification is 
% already in use, different values of unitlength are being used.
% (Magnification 1 seems to always use 3947sp for \unitlength.)
% To implement a uniform magnification change (e.g., 2x for overheads),
% probably the easiest thing to do is to write a perl script which
% modifies *.pstex_t by adding [scale=2] after \includegraphics
% and doubling the unitlength value.  See /k/p/t/magfig.

%Until 2020-08-02:
\usepackage{color} % xfig figures exporting pstex_t files need this
%\usepackage{xcolor} % /w/spectilt/

\newcommand{\myTexFigure}[2]{\myTexFigureAny{#1}{pstex_t}{#2}} % default

% Usage: \myTexFigureAny{name}{extension}{caption}
\newcommand{\myTexFigureAny}[3]{% LATEX VERSION
\begin{figure}[htb]
\centering
\centerline{\input{fig/#1.#2}}  % to hack: \includegraphics{fig/#1.pstex}
\mycaptionparlabel{#3}{fig:#1}
\mystrut
\end{figure}
}

