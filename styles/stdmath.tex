\usepackage{amsmath}  % to get professional mode by default
\usepackage{amssymb}  % \circledast - see LaTeX Companion, p. 222
\usepackage{bm} % bold math - ADDED 7/21/04
%\usepackage{mathtime} - dvips needs ``mtex.pfb'' first

% Upper case preferred (easier to see)
\newcommand{\BEQ}{\[}
\newcommand{\EEQ}{\]}
\newcommand{\BEQN}{\begin{equation}}
\newcommand{\EEQN}{\end{equation}{}} % Extra empty box to work around l2h bug deleting \n afterwards when comment follows
\newcommand{\BEQL}{\begin{equation}}
\newcommand{\EEQL}{\end{equation}{}}
\newcommand{\BEQA}{\begin{eqnarray}}
\newcommand{\EEQA}{\end{eqnarray}{}}
\newcommand{\bea}{\begin{eqnarray}}
\newcommand{\eea}{\end{eqnarray}{}}
\newcommand{\beq}{\[}
\newcommand{\eeq}{\]}
\newcommand{\beqn}{\begin{equation} }
\newcommand{\eeqn}{\end{equation}}
\newcommand{\beqa}{\begin{eqnarray}}
\newcommand{\eeqa}{\end{eqnarray}{}}
\newcommand{\BEAS}{\begin{eqnarray*}}
\newcommand{\EEAS}{\end{eqnarray*}}
\newcommand{\beas}{\BEAS}
\newcommand{\eeas}{\EEAS{}}
% Use BEASMI/EEASMI to force image generation (e.g., for spacing issues)
\newcommand{\BEASMI}{\latexhtml{}{\par\begin{center}\begin{makeimage}}\begin{eqnarray*}}
\newcommand{\EEASMI}{\end{eqnarray*}\latexhtml{}{\end{makeimage}\end{center}\par}}

\newcommand{\BA}{\begin{array}}
\newcommand{\EA}{\end{array}}

% conflicts with /k/l/sm/macros.tex: \ea
%\newcommand{\ba}{\begin{array}}
%\newcommand{\ea}{\end{array}}

% CMJ style \newcommand{\eq}[1]{Eq.~(\ref{eq:#1})}
\newcommand{\eq}[1]{(\ref{eq:#1})} % IEEE style
\newcommand{\Eq}[1]{Equation (\ref{eq:#1})}
\newcommand{\erefb}[1]{Eq.~(\ref{eq:#1})} % brief
\newcommand{\eqn}[1]{(\ref{eq:#1})}
\newcommand{\Eqn}[1]{Equation (\ref{eq:#1})}
\newcommand{\eqns}[1]{(\ref{#1})}


\newcommand{\defn}{\par\vspace{0.1in}{\bf Definition:\ }}
\newcommand{\defnn}[1]{\par\vspace{0.1in}{\bf Definition (#1):\ }} % name in parens
\newcommand{\Definition}{\par\vspace{0.1in}\textbf{Definition. }}
\newcommand{\thm}{\par\vspace{0.1in}\noindent{\bf Theorem:\ }}
\newcommand{\thml}[1]{\par\vspace{0.1in}{\bf Theorem (#1):\ }}
\newcommand{\Theorem}{\textbf{Theorem. }}
\newcommand{\Property}{\textbf{Property. }}
\newcommand{\corr}{\par\vspace{0.1in}\textbf{Corollary:\ }}
\newcommand{\Corollary}{\textbf{Corollary. }}
\newcommand{\Lemma}{\textbf{Lemma. }}
\newcommand{\Prop}{\textbf{Proposition. }}

%Already defined in cktdefs.tex:
% \newcommand{\eg}{\par\vspace{0.1in}{\bf Example:\ }}
\newcommand{\Example}{\par\vspace{0.1in}\noindent{\bf Example:\ }}
\newcommand{\Examples}{\par\vspace{0.1in}\noindent{\bf Examples:\ }}

%does not work: \newcommand{\TheoremNo}[1]{\textbf{Theorem. \label{thm:#1}} } 
\newcommand{\TheoremNo}[1]{\textbf{Theorem #1.}} % 12/20/2010 simplified version
\newcommand{\PropertyNo}[1]{\textbf{Property #1.}} % new 12/20/2010
\newcommand{\LemmaNo}[1]{\textbf{Lemma. \label{lemma:#1}} }
\newcommand{\CorollaryNo}[1]{\textbf{Corollary. \label{cor:#1}} }

%JOS:5/04: was \newcommand{\pf}{\emph{Proof:\ }}
\newcommand{\pf}{\par\vspace{5pt}\noindent\emph{Proof:\ }}
\newcommand{\pfend}{$\Box$}
\newcommand{\pfendmath}{\Box}
\newcommand{\Proof}{\textbf{Proof. }}
\newcommand{\EndProof}{$\Box$}

%does not work: \newcommand{\thmref}[1]{Thm.~(\ref{thm:#1})}
%\newcommand{\thmref}[1]{Prop.{} #1} % 12/20/2010 simplified version
\newcommand{\thmref}[1]{Thm.{} #1} % 12/20/2010 simplified version
\newcommand{\propref}[1]{Property #1} % new 12/20/2010
%\newcommand{\Thmref}[1]{Theorem~(\ref{thm:#1})}
\newcommand{\Thmref}[1]{Theorem #1}

\newcommand{\order}[1]{\Oscr\left(#1\right)}

\newcommand{\lref}[1]{Lemma~(\ref{lemma:#1})}
\newcommand{\Lref}[1]{Lemma~(\ref{lemma:#1})}

\newcommand{\dref}[1]{Def.~\ref{def:#1}}
\newcommand{\Dref}[1]{Definition~\ref{def:#1}}

\newcommand{\epg}{}
\newcommand{\crg}{\\}
\newcommand{\crgg}{\\ \\}

\newcommand{\floor}[1]{\left\lfloor #1\right\rfloor}
\newcommand{\ceil}[1]{\left\lceil #1\right\rceil}
\newcommand{\floortext}[1]{\lfloor#1\rfloor}
\newcommand{\ceiltext}[1]{\lceil{#1}\rceil}
\newcommand{\approaches}{\to}

\DeclareMathOperator{\arccosh}{arccosh}
\DeclareMathOperator{\arcsinh}{arcsinh}

\newcommand{\Ascr}{{\cal A}}
\newcommand{\Bscr}{{\cal B}}
\newcommand{\Cscr}{{\cal C}}
\newcommand{\Dscr}{{\cal D}}
\newcommand{\Escr}{{\cal E}}
\newcommand{\Fscr}{{\cal F}}
\newcommand{\Hscr}{{\cal H}}
\newcommand{\Iscr}{{\cal I}}
\newcommand{\Lscr}{{\cal L}}
\newcommand{\Laplace}{{\cal L}}
\newcommand{\Nscr}{{\cal N}}
\newcommand{\Oscr}{{\cal O}}
\newcommand{\Pscr}{{\cal P}}
\newcommand{\projop}{{\bf P}}
\newcommand{\Rscr}{{\cal R}}
\newcommand{\Sscr}{{\cal S}}
\newcommand{\Tscr}{{\cal T}}
\newcommand{\Xscr}{{\cal X}}
\newcommand{\Zscr}{{\cal Z}}

\newcommand{\ip}[1]{\left<#1\right>}
\newcommand{\iptext}[1]{\langle #1\rangle}
%old version: \newcommand{\ip}[2]{\left<#1,#2\right>}
\newcommand{\normtext}[1]{ ||\,#1\,|| }
\newcommand{\norm}[1]{\left\|\,#1\,\right\|}
\newcommand{\normi}[1]{\norm{#1}_\infty}
\newcommand{\set}[1]{\left\{#1\right\}}

% very common operators -- more in operators.tex
\newcommand{\realpart}[1]{\mbox{re\ensuremath{\left\{#1\right\}}}}
\newcommand{\imagpart}[1]{\mbox{im\ensuremath{\left\{#1\right\}}}}
\newcommand{\realparttext}[1]{\mbox{re\ensuremath{\{#1\}}}}
\newcommand{\imagparttext}[1]{\mbox{im\ensuremath{\{#1\}}}}
\newcommand{\realPart}[1]{\mbox{re}\left\{#1\right\}}
\newcommand{\realPartSq}[1]{\mbox{re}^2\left\{#1\right\}}
\newcommand{\real}{\Re}
\newcommand{\imagPart}[1]{\mbox{im}\left\{#1\right\}}
\newcommand{\imagPartSq}[1]{\mbox{im}^2\left\{#1\right\}}
%defined in amsmath, but I don't know how to get spacing right: 
%FIXME: Emit error when #1 is empty or otherwise odd (catch old usage)
\renewcommand{\mod}[1]{\lp\mbox{mod}\;#1\rp}
\newcommand{\sinc}{\mbox{sinc}}
\newcommand{\diag}{\mbox{diag}}

%defined in amsmath: \newcommand{\implies}{\Rightarrow}
\newcommand{\impliess}{\,\,\Rightarrow\,\,}
\newcommand{\impliesq}{\implies\quad}
\newcommand{\simpliess}{\quad\Rightarrow\quad}

\newcommand{\goesto}{\to}
%\newcommand{\goestoas}[1]{\to_{#1}}
%\newcommand{\goestoas}[1]{\stackrel{\to}{{#1}}}
%\newcommand{\goestoas}[1]{\buildrel{\to}\over{#1}}
\newcommand{\goestoas}[1]{\;\stackrel{\longrightarrow}{{\scriptscriptstyle #1}}\;}
\newcommand{\conj}[1]{\overline{#1}}
% \newcommand{\abstext}[1]{\leftv{#1}\rightv}
\newcommand{\abstext}[1]{|#1|}
% \newcommand{\abs}[1]{\left|\,#1\,\right|}
\newcommand{\abs}[1]{\left|#1\right|}

% moved to stdcommon.tex
% \newcommand{\isdef}{\stackrel{{\scriptscriptstyle\Delta}}{=}}
% \newcommand{\isdeftext}{\buildrel {\scriptscriptstyle\Delta}\over=}
%\newcommand{\isdef}{\buildrel \Delta\over=}

\newcommand{\nn}{\nonumber}
\newcommand{\nnr}{\nonumber \\}

% NOTE: DO NOT end these macros with a newline.
% In that case, a blank line is created between matrices in images.tex 
% unless successive matrices are all on the same line in the source.
% This will cause a failure in l2h output.
\newcommand{\twobytwonp}[4]{\begin{array}{cc} #1 & #2 \\[2pt] #3 & #4 \end{array}}
\newcommand{\twobytwonprs}[5]{\begin{array}{cc} #1 & #2 \\[#5] #3 & #4 \end{array}}
\newcommand{\twobytwo}[4]{\left[\begin{array}{cc} #1 & #2 \\[2pt] #3 & #4 \end{array}\right]}
\newcommand{\twobytwors}[5]{\left[\begin{array}{cc} #1 & #2 \\[#5] #3 & #4 \end{array}\right]}
\newcommand{\twobytwor}[4]{\left[\begin{array}{rr} #1 & #2 \\[2pt] #3 & #4 \end{array}\right]}
\newcommand{\twobyonenp}[2]{\begin{array}{c} #1 \\[2pt] #2 \end{array}}
\newcommand{\twobyone}[2]{\left[\twobyonenp{#1}{#2}\right]}
\newcommand{\onebytwo}[2]{\left[\begin{array}{cc} #1 & #2 \end{array}\right]}
\newcommand{\onebytwotop}[2]{\begin{array}{r}\onebytwo{#1}{#2}\\[2pt]{}\end{array}}
\newcommand{\onebythree}[3]{\left[\begin{array}{ccc} #1 & #2 & #3\end{array}\right]}
\newcommand{\onebythreetop}[3]{\begin{array}{r}\onebythree{#1}{#2}{#3}\\[2pt]{}\\[2pt]{}\end{array}}
\newcommand{\threebyonenp}[3]{\begin{array}{c} #1 \\[2pt] #2 \\[2pt] #3\end{array}}
\newcommand{\fourbyonenp}[4]{\begin{array}{c} #1 \\[2pt] #2 \\[2pt] #3 \\[2pt] #4\end{array}}
\newcommand{\fivebyonenp}[5]{\begin{array}{c} #1 \\[2pt] #2 \\[2pt] #3 \\[2pt] #4\\[2pt] #5\end{array}}
\newcommand{\onebyfour}[4]{\left[\begin{array}{cccc} #1 & #2 & #3 & #4 \end{array}\right]}
\newcommand{\onebyfournts}[4]{\left[\begin{array}{cccc}#1\!&\!#2\!&\!#3\!&\!#4\end{array}\right]}

\newcommand{\twobyonep}[2]{\left(\twobyonenp{#1}{#2}\right)}
\newcommand{\threebyone}[3]{\left[\threebyonenp{#1}{#2}{#3}\right]}
\newcommand{\fourbyone}[4]{\left[\fourbyonenp{#1}{#2}{#3}{#4}\right]}
\newcommand{\fivebyone}[5]{\left[\fivebyonenp{#1}{#2}{#3}{#4}{#5}\right]}
\newcommand{\fivebyonens}[5]{\left[\!\!\fivebyonenp{#1}{#2}{#3}{#4}{#5}\!\!\right]}
\newcommand{\sixbyone}[6]{\left[\sixbyonenp{#1}{#2}{#3}{#4}{#5}{#6}\right]}

\newcommand{\threebythreenpmi}[9]{%
\begin{makeimage}
\threebythreenp{#1}{#2}{#3}{#4}{#5}{#6}{#7}{#8}{#9}
\end{makeimage}
}

\newcommand{\threebythreenp}[9]{%
\begin{array}{ccc} 
#1 & #2 & #3\\[2pt] 
#4 & #5 & #6\\[2pt] 
#7 & #8 & #9 
\end{array}
}

\newcommand{\threebythree}[9]{\left[\threebythreenp{#1}{#2}{#3}{#4}{#5}{#6}{#7}{#8}{#9}\right]}

\newcommand{\threebythreert}[9]{\left[\begin{array}{rrr} 
#1 & #2 & #3\\[2pt] 
#4 & #5 & #6\\[2pt] 
#7 & #8 & #9 
\end{array}\right]}

\newcommand{\twobythree}[6]{\left[\begin{array}{ccc} 
#1 & #2 & #3\\[5pt] 
#4 & #5 & #6
\end{array}\right]}

\newcommand{\twobythreesp}[7]{\left[\begin{array}{ccc} 
#1 & #2 & #3\\[#7pt] 
#4 & #5 & #6
\end{array}\right]}

%in amsmath? 
\providecommand{\degrees}{\mbox{${}^{\circ}$}}

\newcommand{\eqs}{\,\mathrel{\mathop=}\,}
\providecommand{\approxs}{\,\;\approx\;\,}

\newcommand{\lp}{\left(}
\newcommand{\rp}{\right)}

\newcommand{\bl}{\left(} % used? check source bag for \\bl\b
\newcommand{\br}{\right)} % used?

\newcommand{\lb}{\left[}
\newcommand{\rb}{\right]}

\newcommand{\lbr}{\left\{}
\newcommand{\rbr}{\right\}}

% x = \funcalign{foo}{n=0}{bar}{n \neq 0}
\newcommand{\funcalign}[4]{\left\{\begin{array}{ll}
	#1, & #2 \\[5pt]
	#3, & #4 \\
	\end{array}
	\right.}

\newcommand{\funcalignthree}[6]{\left\{\begin{array}{ll}
	#1, & #2 \\[5pt]
	#3, & #4 \\[5pt]
	#5, & #6 \\
	\end{array}
	\right.}

\newcommand{\cpile}[1]{\matrix{#1}}

\newcommand{\conv}{\ast}
\newcommand{\circonv}{\circledast} % AMS: \usepackage{amssymb}

\newcommand{\E}[1]{\Escr\left\{#1\right\}} % expected value

% I would prefer subscripts to superscripts.  Check favorite refs on this point:
%Both in math mode no matter what I try - ANOTHER PERL 5 INTRODUCED BUG LIKELY:
%\newcommand{\Linf}{\latexhtml{{\ensuremath{L^\infty}}}{{$L$-infinity}}}
%\newcommand{\Lp}{\latexhtml{{\ensuremath{L^p}}}{\mbox{Lp}}}
%\newcommand{\Ltwo}{\latexhtml{{\ensuremath{L^2}}}{\mbox{L2}}}
%\newcommand{\Ltwotext}{\latexhtml{{\ensuremath{L^2}}}{\mbox{L2}}}
%\newcommand{\Lone}{\latexhtml{{\ensuremath{L^1}}}{\mbox{L1}}}

\newcommand{\Linf}{\ensuremath{L_\infty}}
\newcommand{\Lp}{\ensuremath{L_p}}
\newcommand{\Ltwo}{\ensuremath{L_2}}
\newcommand{\Ltwotext}{\Ltwo}
\newcommand{\Lone}{\ensuremath{L_1}}

\newcommand{\ldotss}{\ldots\,}
\newcommand{\iffs}{\;\Leftrightarrow\;}
\newcommand{\corrto}{\;\longleftrightarrow\;}
\newcommand{\corrtotext}{\;\leftrightarrow\;}

\newcommand{\zt}{{\it z} transform} % Avoid math so auto-link will work
\newcommand{\zts}{{\it z} transform } % Avoid math so auto-link will work
\newcommand{\Zt}[1]{\ensuremath{\pmb{\mathcal{Z}}\{#1\}}} % bold calligraphic Z-transform

% Also in cktdefs.tex
\providecommand{\zbox}[1]{\fbox{$\displaystyle #1$}}
\newcommand{\zboxlt}[1]{\fbox{$\displaystyle #1$}}

% \newcommand{\doublebox}[2][0.75\hsize]{\begin{center}\fbox{\fbox{%
\newcommand{\doublebox}[2][3in]{\begin{center}\fbox{\fbox{%
\begin{minipage}{#1}{#2}\end{minipage} }} \end{center}}

% Usage: 
% \doublebox[0.3\hsize]{
%   This is a big box
%   \barray
%      x &=&  a + \sum x^2 - 2 \\
%        &=&  a - \sum x^3 - 7
%   \earray
% }

\newcommand{\singlebox}[2][3in]{\begin{center}\fbox{%
\begin{minipage}{#1}{\vspace{-0.15in}#2}\end{minipage} } \end{center}}

% Usage: 
% \singlebox {
%    \doublebox[1in]{This is a box within a box}
%    \barray
%       x &=&  a + \sum x^2 - 2 \\
%         &=&  a - \sum x^3 - 7 \\
%         &=& mc^2
%     \earray
% }

% Really one arg and one optional arg.  How to catch?
%\html{\renewcommand{\singlebox}[2]{#2}}

%% \newcommand{\reals}{{\bf R}}
%% \newcommand{\ints}{{\bf Z}}
%% \newcommand{\complex}{{\bf C}}
%% \newcommand{\RN}{{\bf R}^N} % see also cktdefs.tex
%% \newcommand{\Rthree}{{\bf R}^3}
%% \newcommand{\CN}{{\bf C}^N}
%% \newcommand{\RMM}{{\bf R}^M} % See stdcommon.tex for \RM = circle-R

\newcommand{\reals}{\mathbb{R}}
\newcommand{\ints}{\mathbb{Z}}
\newcommand{\complex}{\mathbb{C}}
\newcommand{\RN}{\reals^N} % see also cktdefs.tex
\newcommand{\Rthree}{\reals^3}
\newcommand{\CN}{\complex^N}
\newcommand{\RMM}{\reals^M} % See stdcommon.tex for \RM = circle-R

\newcommand{\labeledeq}[2]{\[\begin{tabular*}{\textwidth}{l@{\hspace{1in}}c@{\extracolsep{\fill}}r}
{}&$#1$&(#2)\end{tabular*}\]}
\newcommand{\labeledeqhold}[2]{\[#1\]} % Abort

\newcommand{\labeledeqnoeffectoncentering}[1]{\qquad \makebox[0pt][l]{(#1)}}
% NO GOOD: EMITS A BLANK LINE IN MATH MODE THAT KILLS COMPILE OF images.tex (``! Missing $ inserted.''): 
% \newcommand{\labeledeqnot}[1]{} % Abort
\newcommand{\labeledeqnot}[1]{\qquad\mbox{(#1)}} % Abort

\newcommand{\shah}{\,\raisebox{0.8em}{\rotatebox{-90}{\resizebox{1em}{1em}{\ensuremath{\exists}}}}}

\newenvironment{eqnarrayda}{
\[
\begin{array}{lcr@{\,}c@{\,}l}
}{
\end{array}
\]
}
\newcommand{\beasda}{\begin{eqnarrayda}}
\newcommand{\eeasda}{\end{eqnarrayda}}

\newcommand{\relop}[1]{\,#1\,}

%\newcommand{\intrange}[2]{#1,\ldots,#2} % Standard math notation
\newcommand{\intrange}[2]{#1:#2} % Matlab notation
\newcommand{\intrangethree}[3]{#1, #2, \ldots, #3} % Standard math notation
\newcommand{\intrangefour}[4]{#1, #2, #3, \ldots, #4} % Standard math notation

% /k/l/sasp/statdsp.tex
\newcommand{\mux}{\mu_x}
\newcommand{\muxh}{\hat{\mu}_x}
\newcommand{\Ev}[1]{{\cal E}\left\{#1\right\}}
\newcommand{\Var}[1]{\mbox{Var}\left\{#1\right\}}
\newcommand{\vxh}{\hat{\sigma}_x^2}
\newcommand{\rhxn}{\hat{r}_{x(n)}}
